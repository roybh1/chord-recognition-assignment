\documentclass{article}
\usepackage{graphicx}
\usepackage{amsmath}
\usepackage{hyperref}
\usepackage{listings}

\title{Chord Estimation from Audio using Hidden Markov Models}

\author{
\textbf{Roy Ben Haroosh}\\
Department of Computer Science\\
Your Institution\\
\texttt{roy.benharoosh@post.runi.ac.il}\\
\textbf{Gal Davidi}\\
Department of Computer Science\\
Your Institution\\
\texttt{gal.davidi@post.runi.ac.il}} 


\begin{document}

\maketitle

\begin{abstract}
Automatic chord recognition is a significant task within Music Information Retrieval (MIR). Hidden Markov Models (HMMs) using beat-synchronous chroma features have proven effective for chord estimation. This report explores the combination of Neural Networks (NN) and Hidden Markov Models (HMM) and investigates pre-trained audio embeddings, such as VGGish and L3net, to enhance acoustic modeling for chord recognition.
% TODO: remove the NN part and talk about HMM and OpenL3
\end{abstract}

\section{Introduction}
Automatic chord recognition aims to identify and segment chords within audio recordings automatically. Since 2008, it has been a task in the Music Information Retrieval Evaluation eXchange (MIREX), achieving notable success. Hidden Markov Models (HMM) coupled with chroma vectors have established themselves as effective methods. This work evaluates extensions to classical HMM methods by integrating neural network-based acoustic models and leveraging pre-trained embeddings.

\section{Related Works}
Previous studies have shown HMMs combined with Expectation-Maximization (EM) as robust techniques for chord recognition. Notably, studies like \textit{Chord Segmentation and Recognition using EM-Trained HMM} have highlighted the efficacy of such models. Recent advancements leverage deep neural networks and deep embeddings, such as Deep Belief Networks (DBNs), OpenL3 embeddings, and VGGish features, significantly improving acoustic modeling quality.

\section{Architecture}
% TODO: talk about the OpenL3 features and the architecture of HMM
Our proposed architecture leverages an NN/HMM hybrid approach:
\begin{enumerate}
    \item \textbf{Feature Extraction:} Extract beat-synchronous chroma vectors, potentially augmented by pre-trained embeddings (VGGish/OpenL3).
    \item \textbf{Acoustic Model:} Neural network-based models trained to map audio embeddings to chord probabilities.
    \item \textbf{Temporal Model:} Hidden Markov Model integrates temporal dependencies between chord progressions.
\end{enumerate}

The probability of an observed sequence $O$ given a state sequence $Q$ in an HMM is defined by:
\begin{equation}
P(O|Q,\lambda) = \prod_{t=1}^{T} P(o_t|q_t,\lambda)
\end{equation}

The forward algorithm computes the probability of the partial observation sequence up to time $t$ as:
\begin{equation}
\alpha_t(i) = \left[\sum_{j=1}^{N}\alpha_{t-1}(j)a_{ji}\right] b_i(o_t)
\end{equation}
where $a_{ji}$ represents transition probabilities and $b_i(o_t)$ emission probabilities.

\section{Experiments}
%  TODO: check equations
We evaluated various configurations of acoustic models:
\begin{itemize}
    \item Baseline: Classical HMM with chroma features.
    \item NN-based acoustic models combined with HMM.
    \item Pre-trained embeddings from VGGish and L3net as input features.
\end{itemize}

\subsection{Baseline}

The training process involves several key steps. Initially, labeled chord annotation files and corresponding audio files are loaded, from which beat-synchronous chromagrams are extracted. These chroma vectors represent pitch class intensities aligned with the 12-tone Western dominant scale. Subsequently, we calculate the mean chroma vector and covariance matrix for each chord state, defining the emission probabilities of the Hidden Markov Model (HMM). Formally, the emission probability for an observation $o_t$ given a state $q_t$ is modeled as:
\begin{equation}
b_{q_t}(o_t) = \mathcal{N}(o_t;\mu_{q_t}, \Sigma_{q_t})
\end{equation}
where $\mu_{q_t}$ is the mean vector and $\Sigma_{q_t}$ the covariance matrix for the chord state $q_t$. The transition probabilities between chords, $a_{ij}$, indicating the probability of transitioning from chord state $i$ to state $j$, are computed empirically from the training set and arranged in the transition matrix $A$:
\begin{equation}
a_{ij} = P(q_{t+1}=j|q_t=i)
\end{equation}
The likelihood of an observed chroma sequence $O = \{o_1, o_2, \dots, o_T\}$ given a state sequence $Q = \{q_1, q_2, \dots, q_T\}$ is calculated by:
\begin{equation}
P(O|Q,\lambda) = \prod_{t=1}^{T} b_{q_t}(o_t)
\end{equation}
Finally, we optimize the model parameters using the Expectation-Maximization (EM) algorithm, iteratively maximizing the likelihood of the observed sequences. This involves computing forward probabilities defined recursively as:
\begin{equation}
\alpha_t(i) = \left[\sum_{j=1}^{N}\alpha_{t-1}(j)a_{ji}\right] b_i(o_t)
\end{equation}
These steps collectively refine the HMM to achieve accurate chord recognition.

\subsection{Input data improvement}
In our experiments, we investigated the impact of removing percussive sounds from the audio signal. To accomplish this, we utilized the Librosa library, specifically applying the Harmonic-Percussive Source Separation (HPSS) algorithm. The HPSS method decomposes the audio signal into harmonic and percussive components through the following process: Short-Time Fourier Transform (STFT) $\rightarrow$ HPSS $\rightarrow$ Inverse Short-Time Fourier Transform (ISTFT). Formally, we applied the HPSS function provided by Librosa as follows:
\begin{lstlisting}[language=Python]
y_harmonic, y_percussive = librosa.effects.hpss(y, sr=sr)
\end{lstlisting}
where $y$ represents the original audio signal and $sr$ is the sampling rate. After separation, we utilized only the harmonic component to extract chroma features and subsequently train our model. Experimental results demonstrated a slight improvement in chord recognition accuracy using this preprocessing step.

\subsection{Inference improvement}
% TODO: talk about the polling


\section{Results}
Preliminary results indicate:
\begin{itemize}
    \item Classical HMM with chroma achieved accuracy around 25\% F1-score around 29\%. Precision around 38\% and recall around 25\%.
    \item Data improvement significantly improved accuracy, reaching Y\%.
    \item Polling
    \item Incorporation of pre-trained embeddings (OpenL3) further increased accuracy to approximately Z\%.
\end{itemize}
Detailed results are available in the provided notebook and will be continuously updated.

\section{Training Data}
\label{sec:training_data}

The quality and composition of training data play a crucial role in the performance of our chord recognition system.

\subsection{Dataset Overview}
Our model was trained on the Beatles dataset, a widely used benchmark in Music Information Retrieval (MIR) research. This dataset consists of 180 songs by The Beatles with expert-annotated chord labels aligned with the audio recordings. The annotations provide precise timing information for each chord change throughout the songs.

\subsection{Data Characteristics}
The training data exhibits the following key characteristics:
\begin{itemize}
    \item \textbf{Size:} 180 songs with approximately 10 hours of audio content
    \item \textbf{Features:} Beat-synchronous chromagrams (12-dimensional vectors representing 
      the intensity of each pitch class)
    \item \textbf{Chord vocabulary:} 24 chord types (major and minor triads in all 12 keys)
    \item \textbf{Musical diversity:} Spans the Beatles' career from 1962-1970, covering various 
      musical styles and harmonic complexities
\end{itemize}

\subsection{Preprocessing Steps}
Before training, we applied several preprocessing techniques:
\begin{itemize}
    \item \textbf{Beat tracking:} Audio was segmented according to beat positions to create 
      beat-synchronous features
    \item \textbf{Chroma extraction:} 12-dimensional chroma vectors were computed for each beat 
      segment using librosa
    \item \textbf{Normalization:} Chroma features were normalized to ensure consistent scaling 
      across different audio segments
    \item \textbf{Embedding extraction:} For advanced models, we extracted VGGish and OpenL3 
      embeddings from the audio segments
\end{itemize}

\subsection{Data Splitting}
The dataset was divided using a standard 80-10-10 split for training, validation, and testing. 
This approach ensures that our model evaluation reflects its performance on unseen data while 
maintaining the statistical properties of the original dataset. We ensured that songs from the 
same album were kept in the same split to avoid data leakage.

\section{Conclusion}
Our findings demonstrate the effectiveness of integrating neural network acoustic modeling and pre-trained embeddings with traditional HMMs for chord recognition tasks. Further experimentation and hyperparameter tuning may yield additional improvements.

\section*{Code and Media}
The code for this project is available at: \url{https://github.com/caiomiyashiro/music_and_science/tree/master/Chord\%20Recognition}

Media and additional results can be accessed here: \url{https://drive.google.com/drive/folders/1YmfEPtX_QLlpo0sR0kQwFV4-Lz6Sd01W}

\bibliographystyle{plain}
\bibliography{references}

\end{document}

